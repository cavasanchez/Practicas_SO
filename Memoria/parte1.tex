\chapter{Código}
\section{Funciones Auxiliares}
Antes de explicar las funciones pedidas, comentaremos brevemente las funciones auxiliares que hemos utilizados
En primer lugar explicaremos algunas funciones auxiliares que hemos utilizado: 
\subsubsection{char** reservarEspacio(int)}
Esta función se encarga de reservar el espacio necesario para que nuestro array de soluciones tenga el tamaño justo, y cada elemento de dicho array pueda almacenar cualquier String que se introduzca por la entrada.
\subsubsection{void liberarEspacio(char**,int)}
Esta función se encarga de liberar la memoria dinámica tras su uso. Recibe el array de punteros donde se guarda nuestra solucion y le aplica $free$ a todos sus elementos.
\section{Función head}
La primera función a implementar es $int head(int N)$, que deberá comportarse como \textit{head(1)} y devolver las N primeras lineas en la salida estándar recibidas por entrada estándar

\section{Función tail}


\section{Función longLines}
\chapter{Comentarios personales}